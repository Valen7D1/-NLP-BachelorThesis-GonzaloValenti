Sexism is a current problem in society, with a strong presence in social media where the pervasive influence of sexism perpetuates gender roles and leads to systematic harassment of individuals, especially women, based on their gender. Specifically, it is believed that the use of artificial intelligence can help alleviate this problem by employing Natural Language Processing techniques.

EXIST 2023 proposes a competition that invites participants to combat sexism in social media using artificial intelligence, along with a dataset created by the organizers comprising texts extracted from Twitter regarding sexist situations. This competition, now in its third edition, has gained momentum over the years, especially through the use of Transformer-based systems. Specifically, this paper aims to develop two systems to address the first two tasks proposed by the competition: sexism detection and the classification of intentions within sexist tweets as direct, reported, or judgmental.

After generating various models to tackle both tasks, an analysis of the results will be conducted to determine the best model for each task and examine the current state of Transformer-based models in sexism detection. Specifically, the following Transformers are considered: BERT (cased, uncased, and multilingual), RoBERTa (base and twitter-base-emotion), and XLM-RoBERTa (base and large).

\begin{itemize}
\item Task 1: The best model for English-only tweets was generated using the RoBERTa variant, Twitter\_roberta\_base\_emotion, achieving an F1 score of 0.81. For the multilingual format of the task, the best model was Bert multilingual uncased with an F1 score of 0.78.
\item Task 2: The best model for English-only tweets was again generated using Twitter\_roberta\_base\_emotion, obtaining an F1 score of 0.5. For the multilingual task, the best model was also Bert\_base\_multilingual-uncased with an F1 score of 0.61.
\end{itemize}

Finally, after conducting the study, a brief analysis will be provided on the importance of these models in society, as well as the legislative safeguards in place to regulate the use of AI in society.

\textbf{Keywords: Sexism, Exist, Transformer, BERT, F1, natural language processing, Twitter, neural networks}