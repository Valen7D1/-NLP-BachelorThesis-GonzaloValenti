
\section{Impacto}
La detección de sexismo en la sociedad es un tema de gran importancia en la actualidad, ya que el sexismo es un problema social que puede tener consecuencias negativas en la vida de las personas. La inteligencia artificial (IA) y en particular el procesamiento del lenguaje natural (PLN), han demostrado ser herramientas valiosas para detectar el sexismo en el lenguaje utilizado en diferentes contextos, como en redes sociales, textos de noticias \cite{mosqueda2023uso}, anuncios publicitarios \cite{villanueva2013autodependencia}, entre otros.

Un ejemplo concreto de aplicación del NLP para la detección de sexismo en el lenguaje es el análisis de las redes sociales. Las redes sociales son una fuente valiosa de datos para los investigadores que buscan analizar los patrones de lenguaje y comportamiento en línea. El análisis de las redes sociales con herramientas basadas en NLP puede permitir la identificación de patrones discriminatorios en el lenguaje utilizado en las publicaciones, los comentarios y las conversaciones en línea \cite{racca2022deteccion}.

La detección de sexismo basada en NLP también puede ser utilizada en la educación y la concientización sobre el sexismo en la sociedad. Los programas de educación \cite{calero2021modelo} y concienciación pueden utilizar el análisis de patrones discriminatorios en el lenguaje para enseñar a los estudiantes sobre los estereotipos de género y los roles tradicionales de género \cite{azpillaga2021analisis}. También se pueden utilizar herramientas de detección de sexismo basadas en NLP para evaluar la efectividad de los programas de educación y concientización sobre el sexismo.

En conclusión de cara al impacto, el uso de NLP para la detección de sexismo en la sociedad es una herramienta valiosa para combatir el sexismo en el lenguaje y en la cultura en general. Con el desarrollo continuo de las herramientas de NLP y la inteligencia artificial en general, es probable que la detección de sexismo en la sociedad siga mejorando en el futuro.

\section{Legislación}
Desde una perspectiva legislativa y legal, hay varios marcos regulatorios que buscan garantizar que la aplicación de la inteligencia artificial (IA) no viole ninguna ley fundamental y que se respeten los derechos de la sociedad.

En la Unión Europea, por ejemplo, se ha establecido el Reglamento General de Protección de Datos (RGPD \cite{de2016reglamento}), que establece las normas para la recopilación, el procesamiento y el almacenamiento de datos personales. El RGPD establece que cualquier uso de datos personales debe ser justo, transparente y basado en el consentimiento del usuario, y que los usuarios tienen derecho a solicitar que se borren sus datos personales en cualquier momento. Además, la UE está trabajando en un marco regulador específico para la IA \cite{EU-IA} , que se espera que establezca directrices sobre la transparencia y la responsabilidad en la toma de decisiones automatizada.

Por otro lado, España no se queda en absoluto atrás. A efectos de la ley desde el 13 de Julio de 2022 se estableció la ley integral para la igualdad de trato y la no discriminación \cite{leyespanola}. La cual establece que cualquier decisión basada en IA deberá ser auditable, revisable y controvertible por la decisión humana así como garantizar que su uso será en pos del bien común sin tener ningun otro objetivo que no sea ese. Si bien esta ultima linea es algo genérica, cada vez más la ley está garantizando el uso responsable y adecuado de la IA en nuestra sociedad.

Desde una perspectiva legislativa, en conclusión, existen marcos legales y regulatorios que buscan garantizar que la aplicación de la IA respete los derechos de la sociedad y no viole ninguna ley fundamental. Es importante que se sigan desarrollando y actualizando estos marcos para garantizar que la tecnología se utilice de manera ética y responsable.